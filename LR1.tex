\documentclass[14pt]{extreport}
\usepackage{fontspec}
\usepackage{polyglossia}
\usepackage{graphicx}
\setmainlanguage{russian}
\setotherlanguages{english}
\setmainfont{Times New Roman}

% Настройка шрифтов с локальными шрифтами 
% \setmainfont[
% Path=./fonts/,
% Extension=.ttf,
% Ligatures=TeX,
% UprightFont={*-Regular},
% BoldFont={*-Bold},
% ItalicFont={*-Italic},
% BoldItalicFont={*-BoldItalic}
% ]{BookmanOldStyle}


\usepackage[a4paper,left=30mm,right=10mm,top=20mm,bottom=20mm]{geometry}
\setlength{\parindent}{1.25cm}
\usepackage{indentfirst}
\setlength{\parskip}{0pt}
\linespread{1.5}

% Нумерация страниц внизу в центре, исключая титульную страницу
\usepackage{fancyhdr}
\pagestyle{fancy}
\fancyhf{}
\fancyfoot[C]{\thepage}
\fancypagestyle{plain}{%
	\fancyhf{}
	\fancyfoot[C]{\thepage}
}

\title{Реферат на тему: \\[0.5cm] \textbf{Название вашей темы}}
\author{Студент: Зубов Михаил Геннадьевич \\ Группа: ИВТ 1-1}
\date{\today}

\begin{document}
	
	\maketitle
	\tableofcontents
	
	\chapter{Введение}
	Здесь начинается текст введения.
	
	\chapter{Основная часть - 1}
	\section{Раздел 1.1}
	Текст первого раздела.
	
	\subsection{Подраздел 1.1.1}
	Текст первого подраздела.
	
	\subsection{Подраздел 1.1.2}
	Текст второго подраздела.
	
	\subsubsection{Подподраздел 1.1.2.1}
	Текст второго подраздела.
	
	\section{Раздел 2}
	Текст второго раздела.
	
	\chapter{Заключение}
	Здесь подводятся итоги.
	
	% Пример жирного, курсивного и подчёркнутого текста
	\textbf{Жирный текст} может быть полезен для выделения важных терминов. 
	\textit{Курсивный текст} используется для выделения цитат или названий. 
	\underline{Подчёркнутый текст} применяют реже, но он тоже доступен.
	
	\section{Список литературы}
	\begin{enumerate}
		\item Автор1. Название книги. Издательство, год.
		\item Автор2. Название статьи. Журнал, год.
	\end{enumerate}
	
	% Вставка изображения
	\begin{figure}[h]
		\centering
		\includegraphics[width=0.5\textwidth]{C:/Users/skynet/Desktop/2.jpg}
		\caption{Описание изображения}
		\label{fig:example}
	\end{figure}
	
	% Пример таблицы
	\begin{table}[h]
		\centering
		\begin{tabular}{|c|c|c|}
			\hline
			Столбец 1 & Столбец 2 & Столбец 3 \\
			\hline
			Данные 1 & Данные 2 & Данные 3 \\
			\hline
		\end{tabular}
		\caption{Пример таблицы}
		\label{tab:example}
	\end{table}
	
	% Пример формулы
	\begin{equation}
		E = mc^2
	\end{equation}
	
\end{document}
